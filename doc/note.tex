\documentclass[a4paper,10pt,twocolumn,hyperref]{ctexart}
\usepackage{xcolor}
\usepackage{graphicx}
\usepackage{amsmath}
\usepackage{setspace}
\usepackage{bookmark}
\usepackage{textcomp} %提供\textdegree
\usepackage[text={6.5in,9in},centering]{geometry}
\usepackage{array}
\usepackage{multirow}
\usepackage{ulem}
\usepackage{bm}
\usepackage{booktabs} % referred from <Latex_Cookbook>

\ctexset{today=big}


% 一些用于给字体增加颜色的命令
\newcommand{\RT}[1]{\textcolor{red}{#1}}
\newcommand{\BT}[1]{\textcolor{blue}{#1}}
\newcommand{\GT}[1]{\textcolor{green}{#1}}
\newcommand{\MT}[1]{\textcolor{magenta}{#1}}


%%%%%%%%%%%%%%%%%%%%%%%%%%%%%%%%%%%%%%%%%%%%%%%%%%%%%%%%%%%%
\begin{document}
\title{检验巡天编排模拟结果}

\author{许优华~\footnote{\url{yhxu@nao.cas.cn}}}

%\email{yhxu@nao.cas.cn}
%\affiliation{国家天文台}

\maketitle
%%%%%%%%%%%%%%%%%%%%%%%%%%%%%%%%%%%%%%%%%%%%%%%%%%%%%%%%%%%%

\abstract{
本文记录了对“巡天编排模拟”软件的结果的检验。
}

\section{科学需求对观测条件的要求}
% 为了顺利完成所制定的科学目标,望远镜所获取的观测数据的质量需要达到一定的要求。这些具体的要求可以
% 利用Fisher矩阵等工具进行估计,而这样的估计中通常需要假设“已获取”了某种数量和信噪比的观测数据。
% 最终的科学产出可以看作是“数量”与“信噪比”的一个函数,因而可以根据科学目标确定出“数量”与“信噪比”
% 的最低要求。从这些要求出发,可以进一步反推出在什么样的条件下望远镜可以顺利地完成任务。简单说来,
% 就是要排除掉所有不利于观测的情况,使得数据的“信噪比”达到要求,同时又可以获得足够多的观测数据。

根据科学目标反推观测数据所需要达到的要求,以及观测过程中所需要满足的各项条件。

\subsection{CSS-OS的主要科学目标}

\section{各种影响因素的计算和条件判断}
影响观测的主要因素有以下几项:1)各种光学污染,例如太阳光、月球反射光等进入望远镜镜筒后会降低成像
的质量;2)能源因素,如果望远镜长时间处于一种不合理的观测姿态,太阳能帆板无法充分发电,会导致电池
的电量持续下降,进而导致望远镜无法正常工作。

\subsection{太阳}

\subsection{地球}

\subsection{月球遮挡、反光}


\end{document}
